\documentclass[12pt]{article}
\usepackage{hyperref}
\title{The Variation Toolkit}
\author{Pierre Lindenbaum PhD.}
\date{October 11, 2011}

\begin{document}
\maketitle
\section{Introduction}
The Variation Toolkit is a set of C/C++ programs to handle Variant Call Format (VCF).
The programs have been preliminary designed for knime4bio ( \url{http://code.google.com/p/knime4bio/})
but here, the bioinformaticians are the preliminary audience for this toolkit.
\section{Building}
\subsection{Dependencies}
\begin{description}
\item[wget]: wget is used to download the sources for samtools and tabix.
Your web proxy should allow an access to sourceforge.net .
\item[mysql-dev]: the files and libraries for mysql.
\item[libxml2]: the C library for xml \url{http://xmlsoft.org/}.
\item[libxslt]: the C library for xslt \url{http://xmlsoft.org/XSLT/}.
\item[libcurl]: the C library for downloading some URLs.
\end{description}

\section{Tools}

\subsection{scanvcf}
Reads a list of pairs Sample-Name/Path to VCF and prints the VCF, adding an extra column with the sample name to the output.
\begin{quote}
\begin{verbatim}
$ head -n3 input.txt

#Sample	VCF
Sample1	data/sample1.vcf.gz
Sample2	data/sample1.vcf.gz
Sample2	data/sample1.vcf.gz


$ cat input.txt |scanvcf 

#CHROM POS ID REF ALT QUAL FILTER . FORMAT Call SAMPLE
1 879317 rs7523549 C T 71 0 . GT:GQ:DP:FLT 0/1:34:8:0 Sample1
1 880238 rs3748592 A G 51 0 . GT:GQ:DP:FLT 1/1:51:8:0 Sample1
1 880390 rs3748593 C A 99 0 . GT:GQ:DP:FLT 1/0:99:30:0 Sample1
1 881627 rs2272757 G A 99 0 . GT:GQ:DP:FLT 1/0:59:20:0 Sample1
(...)
Y 13524507 . C T 99 0 . GT:GQ:DP:FLT 1/1:99:233:0 Sample20
Y 21154323 rs10465459 G A 99 0 . GT:GQ:DP:FLT 1/1:99:215:0 Sample20
Y 21154426 rs52812045 G A 99 0 . GT:GQ:DP:FLT 1/0:99:143:0 Sample20
Y 21154466 rs10465460 T A 99 0 . GT:GQ:DP:FLT 1/1:99:134:0 Sample20
Y 21154529 . G A 51 0 . GT:GQ:DP:FLT 1/1:51:8:0 Sample20
\end{verbatim}
\end{quote}


\subsection{extractinfo}
Extract a field from the INFO column of a VCF file.

The following script extract the GN(gene name) field from the column INFO. We keep the lines for the gene NOTCH2 and we display the associated SNP.
\begin{quote}
\begin{verbatim}
$ gunzip -c data.vcf.gz | extractinfo -t GN -i | awk -F '       ' '($11 =="NOTCH2")' | cut -d ' ' -f 3 | grep rs

rs6685892
rs2493392
rs2493420
rs7534585
rs7534586
rs2493409
rs2453040
rs2124109

\end{verbatim}
\end{quote}

\subsection{extractformat}
Extract a field from the FORMAT column of a VCF file.

The following command line extract the field 'GT' from the VCF and we count the occurence of the values.

\begin{quote}
\begin{verbatim}
$ gunzip -c data.vcf.gz |\
   extractformat -t GT |\
   cut -d '        ' -f 11 |\
   sort |\
    uniq -c

     29 
  10729 0/1
  10800 1/0
  13518 1/1
     11 1/2
\end{verbatim}
\end{quote}

\subsection{ncbiefetch}
Fetch a record from NCBI database.

Currently supported databases: pubmed.

The following example generates a sequences of 6 pubmed ID and we call ncbiefetch to download the records.
\begin{quote}
\begin{verbatim}
$  (echo "#GI"; seq 1000 2 1010)   |\
      ncbiefetch -c 1 |\
      cut -c 1-100

#GI	pubmed.year	pubmed.title	pubmed.journal	pubmed.abstract
1000	1976	The amino acid sequence of Neurospora NADP-specific glutamate dehydrogenase. The tryptic p
1002	1976	The amino acid sequence of Neurospora NADP-specific glutamate dehydrogenase. Peptic and ch
1004	1976	Properties of 5-aminolaevulinate synthetase and its relationship to microsomal mixed-funct
1006	1976	The attachment of glutamine synthetase to brain membranes.	Biochemical medicine	...
1008	1976	Nature and possible origin of human serum ribonuclease.	Biochemical and biophysical resear
1010	1976	Formation of non-amidine products in the chemical modification of horse liver alcohol dehy
\end{verbatim}
\end{quote}


\subsection{samplepersnp}
Appends a column with the number of Samples per Variation.

The following command line scans the VCF, sort the variations by CHROM/POS/REF/ALT/SAMPLE, counts the number of samples/variation 

\begin{quote}
\begin{verbatim}
$ cat list.tsv | scanvcf  |\
  sort -t'  ' -k1,1 -k2,2n -k4,4 -k5,5 -k11,11 |\
  samplespersnp --sample 11 | awk '($8=".")'

1 753269 rs61770172 C G 99 0 . GT:GQ:DP:FLT 1/1:99:116:0 Sample16 1
1 753405 rs61770173 C A 99 0 . GT:GQ:DP:FLT 1/1:63:31:0 Sample10 7
1 753405 rs61770173 C A 81 0 . GT:GQ:DP:FLT 1/1:51:19:0 Sample12 7
1 753405 rs61770173 C A 35 0 . GT:GQ:DP:FLT 1/0:35:66:0 Sample19 7
1 753405 rs61770173 C A 99 0 . GT:GQ:DP:FLT 1/1:99:35:0 Sample3 7
1 753405 rs61770173 C A 90 0 . GT:GQ:DP:FLT 1/1:90:21:0 Sample5 7
1 753405 rs61770173 C A 99 0 . GT:GQ:DP:FLT 1/1:99:36:0 Sample6 7
1 753405 rs61770173 C A 90 0 . GT:GQ:DP:FLT 1/1:90:21:0 Sample9 7
1 876499 rs4372192 A G 39 0 . GT:GQ:DP:FLT 1/1:39:4:0 Sample12 6
1 876499 rs4372192 A G 42 0 . GT:GQ:DP:FLT 1/1:42:5:0 Sample16 6
1 876499 rs4372192 A G 39 0 . GT:GQ:DP:FLT 1/1:39:4:0 Sample17 6
1 876499 rs4372192 A G 45 0 . GT:GQ:DP:FLT 1/1:45:6:0 Sample18 6
1 876499 rs4372192 A G 45 0 . GT:GQ:DP:FLT 1/1:45:6:0 Sample4 6
1 876499 rs4372192 A G 42 0 . GT:GQ:DP:FLT 1/1:42:5:0 Sample6 6
1 877831 rs6672356 T C 42 0 . GT:GQ:DP:FLT 1/1:42:5:0 Sample14 2
1 877831 rs6672356 T C 39 0 . GT:GQ:DP:FLT 1/1:39:4:0 Sample4 2
1 878601 . C T 98 0 . GT:GQ:DP:FLT 0/1:50:11:0 Sample14 1
\end{verbatim}
\end{quote}

\subsection{numericsplit}
A simple numeric splitter.

The following command line extracts the number of samples/variation and only keep the variation carried by 5 to 9 samples.
\begin{quote}
\begin{verbatim}
$ cat list.tsv | scanvcf  |\
 sort -t'  ' -k1,1 -k2,2n -k4,4 -k5,5 -k11,11 |\
 samplespersnp --sample 11 |\
 numericsplit -c 12 -m 5 -M 9 | awk '($8=".")' | head

1 753405 rs61770173 C A 99 0 . GT:GQ:DP:FLT 1/1:63:31:0 Sample10 7
1 753405 rs61770173 C A 81 0 . GT:GQ:DP:FLT 1/1:51:19:0 Sample12 7
1 753405 rs61770173 C A 35 0 . GT:GQ:DP:FLT 1/0:35:66:0 Sample19 7
1 753405 rs61770173 C A 99 0 . GT:GQ:DP:FLT 1/1:99:35:0 Sample3 7
1 753405 rs61770173 C A 90 0 . GT:GQ:DP:FLT 1/1:90:21:0 Sample5 7
1 753405 rs61770173 C A 99 0 . GT:GQ:DP:FLT 1/1:99:36:0 Sample6 7
1 753405 rs61770173 C A 90 0 . GT:GQ:DP:FLT 1/1:90:21:0 Sample9 7
1 876499 rs4372192 A G 39 0 . GT:GQ:DP:FLT 1/1:39:4:0 Sample12 6
1 876499 rs4372192 A G 42 0 . GT:GQ:DP:FLT 1/1:42:5:0 Sample16 6
1 876499 rs4372192 A G 39 0 . GT:GQ:DP:FLT 1/1:39:4:0 Sample17 6
1 876499 rs4372192 A G 45 0 . GT:GQ:DP:FLT 1/1:45:6:0 Sample18 6
1 876499 rs4372192 A G 45 0 . GT:GQ:DP:FLT 1/1:45:6:0 Sample4 6
1 876499 rs4372192 A G 42 0 . GT:GQ:DP:FLT 1/1:42:5:0 Sample6 6
1 900285 rs4970435 C T 39 0 . GT:GQ:DP:FLT 1/1:39:4:0 Sample11 9
1 900285 rs4970435 C T 42 0 . GT:GQ:DP:FLT 1/1:32:6:0 Sample12 9
1 900285 rs4970435 C T 66 0 . GT:GQ:DP:FLT 1/1:66:13:0 Sample13 9
1 900285 rs4970435 C T 42 0 . GT:GQ:DP:FLT 1/1:42:5:0 Sample14 9
1 900285 rs4970435 C T 48 0 . GT:GQ:DP:FLT 1/1:48:7:0 Sample15 9
1 900285 rs4970435 C T 66 0 . GT:GQ:DP:FLT 1/1:66:13:0 Sample16 9
1 900285 rs4970435 C T 51 0 . GT:GQ:DP:FLT 1/1:51:9:0 Sample17 9


\end{verbatim}
\end{quote}

\subsection{groupbygene}
transposes a VCF table with a "GENE" and a "SAMPLE" column and ouput a new table: count(Gene)=f(SAMPLE)

The following command line extracts the name of the GENE, sort the data on CHROM/POS/REF/ALT/SAMPLE and group the data by gene.

\begin{quote}
\begin{verbatim}
$  cat list.tsv | scanvcf  |\
   extractinfo -t GN | awk '($12!="N/A")' |\
   sort -t '       ' -k1,1 -k2,2n -k4,4 -k5,5 -k11,11 |\
   bin/groupbygene --gene 12 --sample 11


GENE	CHROM	START	END	count(SAMPLES)	count(distinct_MUTATION)	count(Sample1)	count(Sample2)	count(Sample3)	count(Sample4)	count(Sample5)
A1	19	58862835	58864479	5	2	2	2	2	2	2
A1CF	10	52569637	52576068	5	3	1	3	1	1	1
A2M	12	9230038	9264946	5	8	2	2	2	7	3
A2ML1	12	8990937	9020912	5	17	12	7	12	13	10
A4GALT	22	43088971	43089849	5	3	3	3	3	3	1
A4GNT	3	137843106	137850003	4	3	3	3	3	3	0
(...)
\end{verbatim}
\end{quote}

\subsection{normalizechrom}
Normalizes the name of a chromosome to/from UCSC/ENSEMBL.

\begin{quote}
\begin{verbatim}
$ echo -e "1\t10\nX\t20\nMT\t30" | normalizechrom 

chr1	10
chrY	20
chrM	30

$ echo -e "chr1\t10\nX\t20\nchrM\t30" | normalizechrom -E

1	10
X	20
MT	30
\end{verbatim}
\end{quote}

\subsection{ncbiesearch}
Search NCBI/Entrez:

The following example creates a sequence of 3 names, we search the NCBi for each name and the word "Rotavirus" in the title, limit to 2 record, we fetch each record (the PMID is in the 2nd column) and we cut the result down to 80 characters.
 
\begin{quote}
\begin{verbatim}
$ echo -e "#subject\nPiron\nLindenbaum\nPoncet" |\
   ncbiesearch -q '$1 "Rotavirus"[TITL]' -L 2  |\
   ncbiefetch -c 2 |\
   cut -c 1-80
#subject	pubmed.id	pubmed.year	pubmed.title	pubmed.journal	pubmed.abstract
Piron	10888646	2000	Efficient translation of rotavirus mRNA requires simultaneou
Piron	10364288	1999	Identification of the RNA-binding, dimerization, and eIF4GI-
Lindenbaum	15047801	2004	RoXaN, a novel cellular protein containing TPR, LD, and
Lindenbaum	8985320	1997	In vivo and in vitro phosphorylation of rotavirus NSP5 c
Poncet	21864538	2011	Structural Organisation of the Rotavirus Nonstructural Prot
Poncet	20935207	2010	Rapid generation of rotavirus-specific human monoclonal ant
\end{verbatim}
\end{quote}





\subsection{vcfttview}
\begin{quote}
\begin{verbatim}
$ echo -e "ref\t3\nref2\t2" |\
vcfttview -x 3 -B toy.bam -R toy.fa

>ref:3

1         11              21        31         41        51        61           
AGCATGTTAGATAA****GATA**GCTGTGCTAGTAGGCAG*TCAGCGCCATNNNNNNNNNNNNNNNNNNNNNNNNNNNN
      ........    ....  ......K.K......K. ..........                            
      ........AGAG....***...      ,,,,,    ,,,,,,,,,                            
        ......GG**....AA                                                        
        ..C...**** ...**...>>>>>>>>>>>>>>T.....                                 



>ref2:2

1         11            21        31        41        51        61              
aggttttataaaac****aattaagtctacagagcaactacgcgNNNNNNNNNNNNNNNNNNNNNNNNNNNNNNNNNNNN
.............Y    ..W...................                                        
..............****..A...                                                        
 .............****..A...T.                                                      
     .........AAAT.............                                                 
         C...T****....................                                          
           ..T****.....................                                         
             T****......................                                        
                                                               
\end{verbatim}
\end{quote}

\end{document}
